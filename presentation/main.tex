%!TeX program = pdflatex
\documentclass{sptr_presentation}

\title{Introductary Mathematical Physics}
\institute{Spanning Tree}
\author{Marvin Kim}
\date{}

\titlesymbol{\Gamma, J_\alpha, \pi}

\begin{document}

\frame{\titlepage}

\frame{\tableofcontents}

\section{Gamma function}

\subsection{Definition}

\begin{frame}
	\frametitle{Definition of the Gamma Function}

	As you would probably know, the gamma function is an \textbf{extension of the factorial function to the real numbers.} \pause
	It is given by
	\begin{definition}[Gamma Function]
		\begin{equation}
			\Gamma(z) = \int_0^\infty t^{z-1} e^{-t} \, dt
		\end{equation} 
	\end{definition}
	\pause The function is shifted by one from the factorial function because Legendre wanted the pole to be at $0$.
	\begin{equation}
		\Gamma(n) = (n-1)!
	\end{equation}
\end{frame}

\begin{frame}
	\frametitle{Tetsing}

	\begin{definition}
		This is a definition
	\end{definition}

	\begin{proof}
		This is the proof
	\end{proof}
	\begin{theorem}
		This is a theorem
	\end{theorem}

\end{frame}


\end{document}

